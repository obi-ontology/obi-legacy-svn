\documentclass[a4paper]{article}
%\usepackage[latin1]{inputenc}
\usepackage[T1]{fontenc}
\usepackage{times}
\usepackage{url}
%\usepackage{morefloats}
\usepackage{fancyhdr}

\voffset=-30mm
\hoffset=-25mm
\setlength{\textwidth}{170mm}
\setlength{\textheight}{260mm}
%\setlength{\topmargin}{-20pt}
%\setlength{\headheight}{12pt}
%\setlength{\headsep}{35pt}

\pagestyle{empty}

\title{\textbf{OBI: the Ontology for Biomedical Investigations}}
\author{M\'elanie Courtot*  and James Malone*,  on behalf of the OBI Consortium}

\date{}

%% actual document starts here
\begin{document}
%% title information, will be printed later with the \maketitle command 

\maketitle\thispagestyle{empty}              %% makes the title from the information given above (optional)

%%\section{Introduction}        %% document can be divided into several parts with \section \subsection and \subsubsection
 
%%\subsection{Post-Synaptic Signalling}

%% A quote is made with the \cite command. Only papers quoted with \cite will appear in the bibliography

*The poster will be co-presented by these authors.

The Ontology for Biomedical Investigations (OBI)  seeks to provide a cross-domain, shared semantic framework for representing experiment details in the biological and biomedical sciences.  The goal is to develop a broadly-scoped ontology for experiment annotation which supports comparing the provenance of experimental data from different repositories. For instance, to perform credible meta-analysis on gene expression experiments collected using different microarrays and analysis software, one frequently requires more detail than just probe set expression values.  At a minimum, it is desirable to have information about how biological materials were manipulated and how imaging and data reduction techniques were performed. 

The OBI ontology includes generic terms broadly applicable to the description of investigations, and more specific terms, including people involved with an investigation (e.g., IRB member, principal investigator), material (e.g., cDNA library, Epstein-Barr virus transformed B-cell), protocols (e.g., transplantation, cell co-culturing), instrumentation (e.g., flow cytometer, spectrophotometer), and data generated and further analysis (e.g. normalization, visualization).

OBI complements emerging cross-domain minimal information (Minimum Information for Biological and Biomedical Investigations, MIBBI \cite{MIBBI}) and data modeling (Functional Genomics Experiment model, FuGE \cite{FuGE}) standards by rigorously describing the data they contain.
The ontology is under active development by members from 17 distinct technological and biological domains, and is part of the OBO Foundry \cite{Smith}.  It is available in the Web Ontology Language (OWL) format through the National Center for Biomedical Ontology (NCBO) BioPortal \cite{NCBO} or directly at  \url{http://purl.obofoundry.org/obo/obi.owl}. Contributions are welcome via the public mailing list obi-users@groups.google.com or via an issue tracker ( \url{http://purl.obofoundry.org/obo/obi/tracker}).


%% Bibliography 
%%\bibliographystyle{unsrt}                                                       %% defines the bibliography style
\begin{thebibliography}{9}

\bibitem{MIBBI} MIBBI, \emph{The Minimum Information for Biological and Biomedical Investigations}, \url{http://mibbi.sourceforge.net/}.

\bibitem{FuGE} Jones et al., \emph{ The Functional Genomics Experiment model (FuGE): an extensible
    framework for standards in functional genomics}, Nature Biotechnology 25, 1127 - 1133, 2007.

\bibitem{Smith} Smith et al., \emph{The OBO Foundry: coordinated evolution of ontologies to support biomedical data integration}, Nature Biotechnology 25, 1251 - 1255, 2007.


\bibitem{NCBO} NCBO, \emph{The National Center for Biomedical Ontology - BioPortal},\url{http://www.bioontology.org/ncbo/faces/index.xhtml}.

\end{thebibliography}

\end{document}



